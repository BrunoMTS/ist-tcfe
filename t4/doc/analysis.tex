\section{Theoretical Analysis}
\label{sec:analysis}

In this section, we used a suitable theoretical model that was able to estimate the gain and the output and input impedances of the two stages.

\subsection{Gain Stage}

In this subsection we're going to explain how we analysed, theoretically, this stage and, also, how we calculates the important parameters.

\subsubsection{OP Analysis}

In the operating point analysis we used the Thevenin's equivalent to facilitate its analyses. The circuit is shown in the figure below.

\begin{figure}[h] \centering
\includegraphics[width=0.8\linewidth]{slide7.pdf}
\caption{Gain stage circuit (from lecture 17 TCFE) }
\label{fig:rc2a}
\end{figure} 

The important values are computed resorting the following equations:

\begin{equation}
R_B =\frac{R_B1}{R_B1 + R_B2}
\end{equation}

\begin{equation}
- V_eq = \frac{R_B2}{R_B1 + R_B2} \times V_CC
\end{equation}

\begin{equation}
V_eq + R_B \times I_B + V_BEON + R_E \times I_E = 0    
\end{equation}

\begin{equation}
I_E = (1 + \beta_F ) \times I_B    
\end{equation}

\begin{equation}
I_B =\frac{V_eq - V_ON}{R_B + (1 + \beta_F ) \times R_E}    
\end{equation}

\begin{equation}
I_C =\beta_F \times I_B    
\end{equation}

\begin{equation}
V_O = V_CC - R_C \times I_C    
\end{equation}

\begin{equation}
V_E = R_E \times I_E    
\end{equation}

\begin{equation}
V_CE = V_O - V_E    
\end{equation}

Incremental parameters:

\begin{equation}
 g_m = \frac{I_C}{V_T}
\end{equation}

\begin{equation}
r_\pi = \frac{\beta_F}{g_m}    
\end{equation}

\begin{equation}
r_o \approx \frac{V_A}{I_C}    
\end{equation}

with:

\begin{equation}
V_BEON \approx 0.7 V
\end{equation}

\begin{equation}
\beta_F = 178.7    
\end{equation}

\begin{equation}
V_A = 69.7 V    
\end{equation}

The results were the following:

\vspace{0.3in}
\begin{table}[ht]
  \centering
  \begin{tabular}{|l|r|}
    \hline    
    {\bf Name} & {\bf Value [V], [$\Omega$], [A], [S]} \\ \hline
    \input{gsopoctave_tab}
  \end{tabular}
  \caption{Values determined}
  \label{tab:op1a}
\end{table}

\newpage

\subsection{Incremental Analysis}

Through incremental analysis of gain stage, we obtained the following circuit:

\begin{figure}[H] \centering
\includegraphics[width=0.8\linewidth]{slide11.pdf}
\caption{Incremental analysis of gain stage (From lecture 17 TCFE)}
\label{fig:rc2s}
\end{figure} 

In this circuit $R_E$ is canceled by the capacitor bypass. As we are making calculations for our bandwith frequencies, the capacitor already behaves itself like a shortcircuit. We consider $R_E = 0$.

This way, and using the obtained values in the OP analysis, we are able to compute both gain and impedances. 


\begin{equation}
\frac{v_o}{v_i} = R_C \times\frac{-g_m \times r_\pi \times r_o}{(r_o + R_C) \times (R_B || R_\pi)}
\end{equation}

\begin{equation}
   Z_I = R_B || R\pi 
\end{equation}
\begin{equation}
   Z_O = r_o || R_C
\end{equation}

From here we obtained the following values for these parameters:

\vspace{0.3in}
\begin{table}[ht]
  \centering
  \begin{tabular}{|l|r|}
    \hline    
    {\bf Name} & {\bf Value [V] [dB] [$\Omega$]} \\ \hline
    \input{gsoctave_tab}
  \end{tabular}
  \caption{Values of average, maximum and minimum voltage and the ripple value}
  \label{tab:op1s}
\end{table}

\subsection{Output Stage}

In this subsection we're going to explain how we analysed, theoretically, this stage and, also, how we calculated the important parameters as we have done to the gain stage.

\subsubsection{OP Analysis}



\begin{figure}[H] \centering
\includegraphics[width=0.95\linewidth]{Output.pdf}
\caption{Output stage circuit}
\label{fig:rc2d}
\end{figure}


The circuit analysis resulted in the following equations and values:

\begin{equation}
R_E \times I_E +  V_EBON + V_I - V_CC = 0
\end{equation}

\begin{equation}
I_E = \frac{V_CC - V_EBON - V_I}{R_E}    
\end{equation}

\begin{equation}
V_O = V_CC - R_E \times I_E    
\end{equation}

\begin{equation}
V_O = V_I + V_EBON    
\end{equation}

with:

\begin{equation}
V_CC = 12 V
\end{equation}

\begin{equation}
V_EBON \approx 0.7 V    
\end{equation}

\begin{equation}
R_E = 100 \Omega    
\end{equation}

\vspace{0.3in}
\begin{table}[ht]
  \centering
  \begin{tabular}{|l|r|}
    \hline    
    {\bf Name} & {\bf Value [V] [A]} \\ \hline
    \input{osopoctave_tab}
  \end{tabular}
  \caption{OP values computed}
  \label{tab:op1d}
\end{table}

\subsection{Incremental Analysis}

Doing the incremental analysis of the gain stage, we have obtained the following circuit:


\begin{figure}[h] \centering
\includegraphics[width=0.8\linewidth]{slide15.pdf}
\caption{Incremental analysis of de output stage (From lecture 17 TCFE)}
\label{fig:rc2f}
\end{figure} 


The circuit was analyzed in order to obtain the gain and impedances. The equations and values obtained are as follows:

\begin{equation}
    g_\pi= \frac{1}{r_\pi}
\end{equation}

\begin{equation}
    g_E = \frac{1}{R_E}  
\end{equation}

\begin{equation}
    g_o = \frac{1}{r_o}    
\end{equation}

From KCL,

\begin{equation}
    (\frac{1}{R_E} + \frac{1}{r_o}) \times v_o + \frac{v_o - v_i}{r_\pi} - g_m \times v_\pi = 0 
\end{equation}

\begin{equation}
    v_\pi = v_i - v_o    
\end{equation}

which results:

\begin{equation}
    \frac{v_o}{v_i} = \frac{g_m}{g_\pi + g_E + g_o + g_m}
\end{equation}

\begin{equation}
    Z_I = \frac{g_\pi + g_E + g_o + g_m}{g_\pi \times (g_\pi + g_E + g_o)}
\end{equation}

\begin{equation}
    Z_o = \frac{1}{g_\pi + g_E + g_o + g_m}
\end{equation}

\vspace{0.3in}
\begin{table}[ht]
  \centering
  \begin{tabular}{|l|r|}
    \hline    
    {\bf Name} & {\bf Value [V], [dB] and [$\Omega$]} \\ \hline
    \input{osoctave_tab}
  \end{tabular}
  \caption{Values of interest computed}
  \label{tab:op1f}
\end{table}

As we can see, the values go according to the expected, therefore, we have an unitary gain and an output impedance much lower than the speaker 8 $\Omega$.


