\newpage
\section{Simulation Analysis}
\label{sec:simulation}


\subsection{AC/DC Converter}

The objective of this laboratory was to build an AC/DC converter. With this objective in mind, we have implemented the circuit that is represented below (Figure~\ref{fig:acdc}). \\This circuit is constituted by two components, an envelope detector and a voltage regulator, that we are going to analyse in the next subsections.
 
 

\vspace{-0.3in}
\begin{figure}[H] \centering
\includegraphics[width=0.8\linewidth]{acdc.pdf}
\vspace{-0.5in}
\caption{AC/DC Converter developed}
\label{fig:acdc}
\end{figure}

\vspace{-0.85in}
\subsection{Ideal Transformer}

The ideal transformer was introduced in the circuit with two dependent sources, using the relations learned in the theoretical classes, with a $n$ value, that will be balanced to obtain a voltage average of 12V. 


\subsection{Envelope detector}

The purpose of the envelope detector is to convert the AC in a voltage that looks very similar to a DC, with little ripple oscillations. \\This result was obtained using a fullwave rectifier and a capacitor.
In the graphic below (Figure~\ref{fig:dev}), it is shown the output at the terminals of this module.


\vspace{-0.9in}
\begin{figure}[H] \centering
\includegraphics[width=0.45\linewidth]{vedspice.pdf}
\caption{Envelope detector output}
\label{fig:dev}
\end{figure}

\subsection{Voltage Regulator}

 
This module purpose was to attenuate the ripple at the output of the detector envelope in order to improve the parameters. With the implementation of a resistor and 19 diodes in parallel we obtained the following response: 

\vspace{-0.9in}
\begin{figure}[H] \centering
\includegraphics[width=0.45\linewidth]{vospice.pdf}
\caption{Voltage regulator output}
\label{fig:vo}
\end{figure}

\newpage
We can verify and measure the values of the average voltage and of the ripple. This values will be presented in Table~\ref{tab:r}:


\vspace{0.1in}
\begin{table}[ht]
  \centering
  \begin{tabular}{|l|r|}
    \hline    
    {\bf Name} & {\bf Value [V]} \\ \hline
    \input{MAXMIN_tab}
  \end{tabular}
  \caption{Values of average, maximum and minimum voltage and the ripple value}
  \label{tab:r}
\end{table}
\vspace{0.3in}

Finally, we have presented, graphically, $V0-12$ between 50ms and 100 ms in order to verify the behaviour (Fig~\ref{fig:vori}). Also, we reduced the scale to perceive the slight variation in tension.

\vspace{-0.9in}
\begin{figure}[H] \centering
\includegraphics[width=0.6\linewidth]{voripple.pdf}
\caption{Ripple output (vO-12)}
\label{fig:vori}
\end{figure}


\newpage


