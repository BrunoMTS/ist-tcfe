\section{Theoretical Analysis}
\label{sec:analysis}

In this section, the circuit shown in Figure 1 is analysed
theoretically, in terms of its time and frequency responses. 

\subsection{Computation of the voltages in all nodes and currents in all branches for t\textless 0;}

We can apply the node method, that is based in the Kirchhoff's Current Law (KCL) and in the Ohm's Law, to analyse circuits. According to this method, KCL is applied in nodes that aren't connected to voltage sources and the currents flowing into a node must add up to zero, which means that the algebraic sum of currents that enters into a node equals to the algebraic sum of currents that exits that node. In nodes related by voltage sources, additional equations are applied. 
For t\textless0 the capacitor acts like an open circuit, so there is no current flow, in the branch containing the capacitor.\par
\par
So, applying KCL for node 1, we have: 

\begin {equation}
V_1 - V_0= V_s
  \label {eq:kvl}
\end{equation}

Node 2:
\begin {equation}
  (V_3 - V_2)\times G_2 = (V_2 - V_1)\times G_1 + (V_2 - V_5)\times G_3
  \label {eq:kvl}
\end{equation}

Node 3:
\begin{equation}
(V_3- V_2)\times G_2 = K_b\times (V_2 - V_5)
  \label {eq:kvl}
\end{equation}

Node 6:
\begin {equation}
(V_5 - V_6)\times G_5 = K_b\times (V_2 - V_5)
  \label {eq:kvl}
\end{equation}

And node 7:
\begin {equation}
(V_0-V_7)\times G_6 = (V_7 - V_8)\times G_7
  \label {eq:kvl}
\end{equation}

The additional equations are:

\begin {equation}
V_5 - V_8 = -K_d\times (V_7)\times G_6
  \label {eq:kvl}
\end{equation}

\begin {equation}
(V_2 - V_5)\times G_3 - (V_5 - V_0)\times G_4 - (V_5 - V_6)\times G_5 + (V_7 - V_8)\times G7 = 0
  \label {eq:kvl}
\end{equation}

The equation (7) are obtained using the voltage drop between the voltage source. The last equation was derived using a super-node that contains the dependent voltage source. Applying the node law, the sum of currents that enter the dependent voltage source is equal to the sum of currents that leaves it. 

\begin{figure}[h] \centering
\includegraphics[width=0.7\linewidth]{rc2.pdf}
\caption{Circuit that was used to obtained the 2.1 equations.}
\label{fig:rc1}
\end{figure}

The values that where computed from the system of 8 equations are represented in the next table (table \ref{tab:NM}).


\begin{table}[h]
  \centering
  \begin{tabular}{|l|r|}
    \hline    
    {\bf Name} & {\bf Value [A or V]} \\ \hline
    \input{octavevalues1_tab}
  \end{tabular}
  \caption{Theoretical values obtained using octave to solve the equations of Node Method for the expressed conditions. {\em Current}
    is expressed in Ampere and {\it Tensions} are expressed in
    Volt.}
  \label{tab:NM}
\end{table}


\newpage
\subsection{Determination of the equivalent resistance ($Req$) as seen from the capacitor terminals}

As suggested in the laboratory task, we made $V_s$=0 and we replaced the capacitor with a voltage source $V_x$= $V_6$-$V_8$, where $V_6$ and $V_8$ are the voltages in nodes 6 and 8 (as obtained in 2.1). Then we ran a nodal analysis to determine the current $I_x$ supplied by $V_x$. Finally we computed the equivalent resistor as $R_eq = V_x/I_x$, and the time constant.
This procedure is necessary to determine the initial conditions, in this case they correspond to the ones of a saturated capacitor. With the computation of the current that passes through the capacitor, we can obtain the equivalent resistor, which is necessary for the calculation of the time constant, this one will allows us to deduce the natural response of this system. 
Using the node method one more time to analyse the circuit, we will obtain the next series of equations:

\begin {equation}
  (V_3 - V_2)\times G_2 = (V_2 - V_1)\times G_1 + (V_2 - V_5)\times G_3
  \label {eq:kvl}
\end{equation}

Node 3:
\begin{equation}
(V_3- V_2)\times G_2 = K_b\times (V_2 - V_5)
  \label {eq:kvl}
\end{equation}

Node 6:
\begin {equation}
(V_6 - V_8)= V_x
  \label {eq:kvl}
\end{equation}

And node 7:
\begin {equation}
(V_0-V_7)\times G_6 = (V_7 - V_8)\times G_7
  \label {eq:kvl}
\end{equation}

The additional equations are:

\begin {equation}
V_1 - v_0= V_s
  \label {eq:kvl}
\end{equation}

\begin {equation}
V_5 - V_8 = -K_d\times (V_7)\times G_6
  \label {eq:kvl}
\end{equation}

\begin {equation}
  (V_2 - V_5)\times G_3 + (V_7 - V_8)\times G_7 = (V_5 - V_0)\times G_4 + Kb\times(V_2 - V_5)
  \label {eq:kvl}
\end{equation}



The main differences compared to the previous system, are the definition of a super-node that contains the nodes 5, 6 and 8, and the definition of the voltage source. We obtained the next results, from the computation of the above equations.



\begin{table}[h]
  \centering
  \begin{tabular}{|l|r|}
    \hline    
    {\bf Name} & {\bf Value [A, V, $\Omega$ or $s^-1$]} \\ \hline
    \input{octavevalues2_tab}
  \end{tabular}
  \caption{Theoretical values obtained using octave to solve equations of Node Method for the expressed conditions.}
  \label{tab:DS}
\end{table}


\subsection{Computation of the natural solution $v_6n(t)$, in the interval [0, 20] ms}


After we compute the $R_(eq)$ and the current that passes through the capacitor, we can simplify the circuit to a  single V-R-C loop where a current $i(t)$ circulates. The
voltage source $v_I(t)$ drives its input, and the output voltage $v_O(t)$ is taken from
the capacitor terminals. Applying the Kirchhoff Voltage Law (KVL), a single
equation for the single loop in the circuit can be written as

\begin{equation}
  Ri(t) + v_O(t) = v_I(t).
  \label{eq:kvl}
\end{equation}

Because $v_O$ is the voltage between capacitor C's plates, it is related to the
current $i$ by
\begin{equation}
  i(t) = C\frac{dv_O}{dt}.
\end{equation}

Hence, Equation~(\ref{eq:kvl}) can be rewritten as
\begin{equation}
  RC\frac{dv_O}{dt} + v_O(t) = v_I.
  \label{eq:kvl2}
\end{equation}

Equation~(\ref{eq:kvl2}) is a linear differential equation whose solution is a
superposition of a natural solution $v_{On}$ and a forced solution $v_{Of}$:

\begin{equation}
  v_O(t) = v_{On}(t) + v_{Of}(t).
  \label{eq:vo_sol}
\end{equation}

As learned in the theory classes the natural solution can be represented by the next equation:
\begin{equation}
  v_{On}(t) = Ae^{-\frac{t}{RC}},
  \label{eq:vo_nat}
\end{equation}
where $A$ is an integration constant.
As we know the initial conditions, we can obtain the constant $A$ that is going to be equal to $V_x$. In this way, the natural solution is obtained.


With the results obtained in 2.2),  we have determined the natural solution $v_6n(t)$, in the interval [0, 20] ms. It was used the capacitor voltage $V_x$ for t\textless0 as the initial condition.

Using the next equation (knowing $V_x$ and the greek letter tau) we can plot the result, as we can see in Figure \ref{fig:nat}

\begin{equation}
  V_xe^{-\frac{t}{{\displaystyle \tau }}}=V_6n(t),
  \label{eq:vo_nat}
\end{equation}

\begin{figure}[h] \centering
\includegraphics[width=0.8\linewidth]{natural_response.eps}
\caption{Natural response in v6.}
\label{fig:nat}
\end{figure}





\subsection{Computation of the forced solution $v_6n(t)$, in the interval [0, 20] ms}\


To determine the forced solution $v_6f(t)$ in the same interval we, as suggested, used a phasor voltage source $\tilde{V_n}$ , replaced $C$ with its impedance $\tilde{Z_c}$ , and  we ran the previous nodal analysis to determine the phasor voltages in all nodes. 

The next table shows the obtained values.

\begin{table}[h]
  \centering
  \begin{tabular}{|l|r|}
    \hline    
    {\bf Name} & {\bf Value [V]} \\ \hline
    \input{octavevalues3_tab}
  \end{tabular}
  \caption{Theoretical values obtained using octave to solve equations using phasors for the expressed conditions.}
  \label{tab:PH}
\end{table}

\newpage
In this way, as was expected, the only phasor that is not above the real axis, is the one that is not related to the capacitor. 

To determine the expression, $v_6f(t)$, we had to obtain the gain and the phase delay:


\begin{equation}
  \frac{v6}{vs}=|\tilde{V_6}|
  \label{eq:vo_nat}
\end{equation}

\begin{equation}
  {{\displaystyle \phi }}=arg(\tilde{V_6})
  \label{eq:vo_nat}
\end{equation}

\begin{equation}
  {v_6f(t)}=|\tilde{V_6}|*sin(2\pi f+ {\displaystyle \phi })
  \label{eq:vo_nat}
\end{equation}


\begin{figure}[h] \centering
\includegraphics[width=0.8\linewidth]{forced_response.eps}
\caption{Final total solution $v_6(t)$. The orange function corresponds to the forced solution of v6(t) and the blue one corresponds to sinusoidal voltage source }
\label{fig:forced}
\end{figure}
\newpage

\subsection{Computation of the final total solution $v_6(t)$}

To determine the final total solution $v_6(t)$, we converted the phasors to real time functions with $f=1KHz$, and superimposed the natural and forced solutions (as we can see in Figure \ref{fig:tot}). In this picture we plotted $v_s(t)$ and $v_6(t)$ in a time interval of [-5,20]ms.

\begin{equation}
  v_6(t)= V_xe^{-\frac{t}{{\displaystyle \tau }}} + |\tilde{V_6}|*sin(2\pi f+ {\displaystyle \phi })
  \label{eq:vo_nat}
\end{equation}
\begin{figure}[h] \centering
\includegraphics[width=0.8\linewidth]{total_solution.eps}
\caption{Final total solution $v_6(t)$. The Red function corresponds to $v_6(t)$ and the blue one to $v_s(t)$}
\label{fig:tot}
\end{figure}



\subsection{Frequency Responses}
In this part it was asked for us to determine the frequency responses $v_c(f)=v_6(f)-v_8(f)$, and $v_6(f)$ and to plot $v_s(f)$,  $v_c(f)$  and $v_6(f)$ in the same figure. After we have concluded the given task, we can formulate some conclusions- The behaviour of the capacitor is due to the fact that as the frequency applied to the capacitor increases, its reactance decreases (measured in ohms). Likewise as the frequency across the capacitor decreases its reactance value increases. So, $v_c(t)$ is going to tend to zero with the increase of the frequency, because its impedance also tends to zero. That is, we have a short circuit. If the tension $v_c(t)$ starts tending to zero, this one stops of having influence in $v_6(t)$, and this last stays constant. $v_s(t)$ always have a constant value and a constant phase.The Bode's diagram of $v_c(t)$ allow us to conclude that we are in front of a low-pass filter, that is, this filter is a high frequency attenuater.

\begin{figure}[h] \centering
\includegraphics[width=0.8\linewidth]{at51.eps}
\caption{Bode diagram (magnitude in dB's) of $v_6(t)$ }
\label{fig:51}
\includegraphics[width=0.8\linewidth]{at52.eps}
\caption{F{Bode diagram (magnitude in dB's) of $v_6(t)$ and $v_c(t)$}}
\end{figure}

\begin{figure}[h] \centering
\label{fig:52}
\includegraphics[width=0.8\linewidth]{at53.eps}
\caption{{Bode diagram (phase in degrees) of $v_6(t)$ } }
\label{fig:53}
\includegraphics[width=0.8\linewidth]{at54.eps}
\caption{{Bode diagram (magnitude in dB's) of $v_6(t)$, $v_c(t)$ and $v_s(t)$ }}
\label{fig:54}
\end{figure}


\newpage
