\section{Introduction}
\label{sec:introduction}

The objective of this laboratory assignment is to design a Band-Pass filter, using the knowledge  we have acquired in the theoretical classes (lecture 22, 23 and 24). Based on the information given, the Band-Pass filter has to have a Gain of 40dB and to be centred at 1KHz. It receives an alternating current (AC) with 0.01V and a frequency of 1KHz.The primal objective was to create a Band-Pass filter with the highest possible merit figure.



This Merit Figure is given by:
\paragraph{}

\begin{equation}
    M = \frac{1}{Cost (VoltageGainDeviation + CentralFrequencyDeviation + 10^{-6})}  
\end{equation}

\paragraph{}
\begin{itemize}
    \item cost = cost of resistors + cost of capacitors + cost of transistors;
    \item cost of resistors = 1 monetary unit (MU) per kOhm;
    \item cost of capacitors = 1 MU/$\mu$F;
    \item cost of transistors = 0.1 MU per transistor;
\end{itemize}

\paragraph{}
To implement the Band-pass Filter we were only allowed to use the following components:

\begin{itemize}
    \item One 741 OPAMP;
    \item At most three $1 k\Omega$ resistors;
    \item At most three $10 k\Omega$ resistors;
    \item At most three $100 k\Omega$ resistors;
    \item At most three $220 nF$ resistors;
    \item At most three $1 \mu$F resistors;
\end{itemize}

\paragraph{}

In Section~\ref{sec:analysis}, a theoretical analysis of the circuit is
presented. In Section~\ref{sec:simulation}, the circuit is analysed by
simulation, and the results are compared to the theoretical results obtained in
Section~\ref{sec:analysis}. 
In Section~\ref{sec:sidebyside} we did a Side by Side comparison between the simulation and the theoretical results and then we have computed the Figure of Merit as is shown in section~\ref{sec:results}.
The conclusions of this study are outlined in Section~\ref{sec:conclusion}.


